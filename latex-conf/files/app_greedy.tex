\begin{proof}[Proof of Claim~\ref{claim:date}] Since $S \subseteq \NZ(f)$, the
residual $r = f - w$ is such that $\NZ(r) \subseteq \NZ(f)$. Consider $i \in \NZ(r)$
that maximizes $|r_i|\cdot \ltwonorm{e^i}$. Then, we have:
\begin{align}
\left|\frac{\ip{e^i}{r}}{\ltwonorm{e^i}} \right| &\geq
\frac{|r_i|\ip{e^i}{e^i}}{\ltwonorm{e^i}} - \sum_{j \in \NZ(r), j \neq i}
\frac{|r_j|\ip{e^i}{e^j}}{\ltwonorm{e^i}} \nonumber  \\
&\geq |r_i|\cdot \ltwonorm{e^i} - \frac{1}{2k} \sum_{j \in \NZ(r), j \neq i}
|r_j| \cdot \ltwonorm{e^j} \label{eqn:chives} \\
&\geq |r_i| \cdot \ltwonorm{e^i} \cdot \frac{k+1}{2k}, \nonumber
\end{align}
where in the last two steps we used the fact that $\corr(x_i, x_j) =
\ip{e^i}{e^j}/(\ltwonorm{e^i} \ltwonorm{e^j}) \leq 1/(2k)$ and that $|\NZ(r)
\setminus \{ i \}| \leq k-1$. On the other hand, for any $i^\prime \not\in
\NZ(r)$, we have
\begin{align}
\left| \frac{\ip{e^{i^\prime}}{r}}{\ltwonorm{e^{i^\prime}}} \right| &\leq
\sum_{j \in \NZ(r)} \frac{|r_j\ip{e^{i^\prime}}{e^j}|}{\ltwonorm{e^{i^\prime}}}
\nonumber \\
&\leq \frac{1}{2k} \sum_{j \in \NZ(r)} |r_j| \cdot \ltwonorm{e^j} \leq \frac{1}{2}
|r_i| \cdot \ltwonorm{e^i} \label{eqn:dill}
\end{align}

\noindent Here, again in the last two steps, we have used the fact that
$\corr(x_{i^\prime}, x_j) \leq 1/(2k)$ and that $|r_i| \cdot \ltwonorm{e^i}$ is
the largest such term.

First, we claim that if $\ltwonorm{r}^2 \geq \epsilon$, then the $i$ that
maximized $|r_i| \cdot \ltwonorm{e^i}$ must be from the set $\NZ(f) \setminus
S$. (Note that $\ltwonorm{r}^2 = \ltwonorm{f - w}^2 = \loss_{f, D}(w)$, so if
$\ltwonorm{r}^2 \leq \epsilon$, evolution has reached its goal.) By the
triangle inequality, $\sum_{i \in \NZ(r)} |r_i| \cdot \ltwonorm{e^i} \geq
\ltwonorm{r} \geq \sqrt{\epsilon}$. Hence, it must be the case that $|r_i| \cdot
\ltwonorm{e^i} \geq \sqrt{\epsilon}/{k}$. For contradiction, assume that $i \in
S$. Then, since $f^S$ is the projection of $f$ in the space spanned by $S$, we
have $\ip{e^i}{r} = \ip{e^i}{f^S - w}$, since $r = f - f^S + f^S - w$ and
$\ip{e^i}{f - f^S} = 0$. But, by the assumption of the claim, $|\ip{e^i}{r}|\leq
\ltwonorm{e^i} \cdot \ltwonorm{f^S - w} \leq \ltwonorm{e^i} \cdot
\sqrt{\epsilon}/(2k)$, and by (\ref{eqn:chives}), we know that $|\ip{e^i}{r}|
\geq \ltwonorm{e^i} \cdot (|r_i| \cdot \ltwonorm{e^i}) \cdot (k+1)/(2k) >
\ltwonorm{e^i} \cdot \ltwonorm{r} / (2k) \ge \ltwonorm{e^i} \cdot \sqrt{\epsilon}/(2k)$.
Thus, it cannot be the case that $i \in S$.

Let $w^\prime = w + \gamma e^i$. Then, 
%%
\begin{align} 
%
\ltwonorm{f - (w + \gamma e^i)}^2 - \ltwonorm{f - w}^2 &= - 2 \gamma \ip{f -
w}{e^i} + \gamma^2 \ltwonorm{e^i}^2 \nonumber \\
%
&\leq - \ltwonorm{e^i}^2 \left(|\gamma| \cdot |r_i| \cdot \frac{k+1}{k} -
|\gamma|^2\right) \nonumber
%%
\intertext{Now suppose $\gamma$ satisfies $1 - \delta \leq
(2|\gamma|k)/(|r_i|(k+1)) \leq 1 + \delta$, then using the fact that the quadratic
function on the RHS is maximized at $|\gamma| = (k+1) |r_i|/(2k)$, we have}
\ltwonorm{f - (w + \gamma e^i)}^2 - \ltwonorm{f - w}^2 &\leq - \ltwonorm{e^i}^2
\cdot \frac{r_i^2}{4} \cdot \frac{(k+1)^2}{k^2} \cdot (1 - \delta^2) \label{eqn:eggplant}
\end{align}

Note that for any $i^\prime \not\in S$, the ``best'' representation of the form
$w + \beta e^{i^\prime}$ is when $\beta =
\ip{e^{i^\prime}}{r}/\ltwonorm{e^{i^\prime}}^2$, and the corresponding reduction
in squared loss is $(\ip{e^{i^\prime}}{r})^2/\ltwonorm{e^{i^\prime}}^2$. Thus,
for any $i^\prime \neq i$, we have
%%
\begin{align}
%%
\ltwonorm{f - (w + \beta e^{i^\prime})}^2 - \ltwonorm{f - w}^2 &\leq -
\frac{\ip{e^{i^\prime}}{r}^2}{\ltwonorm{e^{i^\prime}}^2} \nonumber \\
%
&\leq - \frac{r_i^2}{4} \ltwonorm{e^i}^2 &\mbox{Using~(\ref{eqn:dill})}
\nonumber
\end{align}
Setting $\delta = \sqrt{1/(k+1)}$ completes the proof of the claim. To
see that $b - a \geq \sqrt{(k+1)\epsilon}/k^2$, notice that any $\gamma$,
such that $|\gamma| \in [(1 - \delta) ((k+1)/(2k)) |r_i|, (1 - \delta)
((k+1)/(2k)) |r_i|]$, achieves the claimed reduction in squared loss. Since
$|r_i| \cdot \ltwonorm{e^i} \geq \sqrt{\epsilon}/k$, we have that $|r_i| \geq
\sqrt{\epsilon}/k$. Hence, $b - a \geq 2 \delta ((k+1)/2k) (\sqrt{\epsilon}/k)
\geq \sqrt{(k+1) \epsilon}/k^2$, for $\delta = \sqrt{1/(k+1)}$.
\end{proof}

\begin{proof}[Proof of Claim~\ref{claim:elderberry}] 
The proof follows along the lines of the proofs of Claims~\ref{claim:apple} and
\ref{claim:banana}.  First, suppose that $\ltwonorm{w} \geq 2 \ltwonorm{f^S}$.
In this case, we claim that for $\gamma \in [1/2, 3/4]$, the mutation $\gamma w$
reduces the squared loss by at least $\ltwonorm{f^S - w}^2/12$. This analysis is
completely identical to that in Claim~\ref{claim:apple} and hence is omitted.
The only difference is that the probability that such a mutation is selected is
$1/16$, conditioned on the event that the mutator chooses mutations that don't
add an extra variable.

Next, we assume that $\ltwonorm{w} \leq 2 \ltwonorm{f^S}$. Let $r^S = f^S - w$.
Now, as in the proof of Claim~\ref{claim:date}, consider $i \in \NZ(r^S)$ (recall
that $\NZ(r^S) = S$) that maximizes $|r^S_i| \cdot \ltwonorm{e^i}$.
Then, the following is true:
\begin{align}
%%
\left| \frac{\ip{e^i}{r^S}}{\ltwonorm{e^i}} \right| &\geq |r^S_i|
\frac{\ip{e^i}{e^i}}{\ltwonorm{e^i}} - \sum_{j \in S, j \neq i} |r^S_j|
\frac{\ip{e^i}{e^j}}{\ltwonorm{e^i}} \nonumber \\
%%
&\geq |r^S_i| \cdot \ltwonorm{e^i} - \frac{1}{2k} \sum_{j \in S, j \neq i}
|r^S_j| \cdot \ltwonorm{e^j} \nonumber \\
%%
&\geq |r^S_i| \cdot \ltwonorm{e^i} \cdot \frac{k+1}{2k} \label{eqn:fennel}
\end{align}
%
where in the last two steps we used the fact that $\corr(x_i, x_j) \leq 1/(2k)$
and that $|r^S_j| \cdot \ltwonorm{e^j}$ is maximized for $j = i$, and the fact
that $|S| \leq k$. Also, by the triangle inequality, we know that $\sum_{j \in
S} |r^S_j| \cdot \ltwonorm{e^j} \geq \ltwonorm{r^S}$; hence, by definition of
$i$, we have that $|r^S_i| \cdot \ltwonorm{e^i} \geq \ltwonorm{r^S}/k$. Now, let
$\beta = \ip{e^i}{r^S}/\ltwonorm{e^i}^2$, and for $\gamma \in [\beta -
|\beta|/2, \beta + |\beta|/2]$, consider the mutation $w + \gamma e^i$. We have,
%%
\begin{align*}
%%
\ltwonorm{f^S - (w + \gamma e^i)}^2 - \ltwonorm{f^S - w}^2 &= -2 \gamma
\ip{e^i}{r^S} + \gamma^2 \ltwonorm{e^i}^2 \\
%%
&= - \ltwonorm{e^i}^2(2 |\gamma| |\beta| - |\gamma|^2) \\
%%
&\leq - \frac{3}{4} \ltwonorm{e^i}^2 \beta^2 %&\mbox{For $\gamma \in [\beta -
%|\beta|/2, \beta + |\beta|/2]$} \\
%%
&\leq - \frac{3}{16} \frac{\ltwonorm{r^S}^2}{k^2} %&\mbox{Using
%(\ref{eqn:fennel}) and the defn. of $\beta$}
\end{align*}

Above $\gamma \in [\beta - |\beta|/2, \beta + |\beta|/2]$. The last step follows
from (\ref{eqn:fennel}) and the definition of $\beta$.
In order for $w + \gamma e^i$ to be a valid mutation, we need to check that
$|w_i| + 3 |\beta|/2 < B$. To see this, observe the following:
%%
\begin{align*}
\ltwonorm{w}^2 &\geq \sum_{j \in \NZ(w)} w_j^2 \ltwonorm{e^j}^2 - \sum_{j_1 < j_2; j_1,
j_2 \in \NZ(w)} 2 |w_{j_1} w_{j_2}| |\ip{e^{j_1}}{e^{j_2}}| \\
&\geq \sum_{j \in \NZ(w)} w_j^2 \ltwonorm{e^j}^2 - \frac{1}{k} \sum_{j_1 < j_2}
|w_{j_1} w_{j_2}| \ltwonorm{e^{j_1}}\ltwonorm{e^{j_2}} \\
&\geq \frac{1}{2} \sum_{j \in \NZ(w)} w_j^2 \ltwonorm{e^j}^2 + \frac{1}{2k} \sum_{j_1 < j_2}
(|w_{j_1}| \ltwonorm{e^{j_1}} - |w_{j_2}| \ltwonorm{e^{j_2}})^2 \\
&\geq \frac{1}{2} \sum_{j \in \NZ(w)} w_j^2 \ltwonorm{e^j}^2 %\qquad \eanote{~Check!}
\end{align*}

\noindent Hence, by a fairly loose analysis, $|w_i|  \leq 2 \ltwonorm{w}/\Delta
\leq 4 \ltwonorm{f}/\Delta \leq 4uk/\Delta$ (since $\ltwonorm{e^i} \geq \Delta$
for the class of distributions defined in Defn.~\ref{defn:bhutan}). Also
$|\beta| = |\ip{e^i}{r^S}|/\ltwonorm{e^i}^2 \leq \ltwonorm{r^S}/\ltwonorm{e^i}
\leq 3 \ltwonorm{f}/\Delta$ (since $\ltwonorm{r^S} \leq \ltwonorm{f^S} +
\ltwonorm{w} \leq 3 \ltwonorm{f^S} \leq 3 \ltwonorm{f}$). It's easy to see that
$|w_i| + (3/2) |\beta| \leq B$, for $B = 10 uk/\Delta$.

Finally, note that conditioned on the mutator choosing a mutation that doesn't
add new variables, the probability of choosing such a mutation is at least
$|\beta|/(8Bk)$ ($1/4$ for choosing a mutation of type ``adjusting'', $1/k$ for
choosing the appropriate variable to adjust and $|\beta|/(2B)$ for choosing the
correct value). Combining the claims for mutations of the ``scaling'' and
``adjusting'' types and taking the appropriate minimum values proves the
statement of the claim.
\end{proof}
