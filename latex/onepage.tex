\documentclass{acmtr2e}

\usepackage{amsmath,amssymb,amsfonts,fullpage}
%\usepackage{hyperref}
\usepackage[usenames]{color}
\usepackage{graphicx}
%\usepackage{subfigure}
%\usepackage[round, semicolon]{natbib}
%%\usepackage[round,colon]{natbib}

\firstfoot{}
\runningfoot{}

% \newcommand{\todovk}[1]{{\marginpar{\color{blue} (VK) #1}}}
% \newcommand{\todoea}[1]{{\marginpar{\color{red} (EA) #1}}}
% \newcommand{\vknote}[1]{{\color{blue} (VK) #1}}
% \newcommand{\eanote}[1]{{\color{red} (EA) #1}}
\newcommand{\todovk}[1]{}
\newcommand{\todoea}[1]{}
% \newcommand{\vknote}[1]{{\color{blue} (VK) #1}}
% \newcommand{\eanote}[1]{{\color{red} (EA) #1}}
\newcommand{\eat}[1]{ }

\def\ie{{\it i.e.},~}
\def\eg{{\it e.g.},~}
\def\etal{{\it et al.}~}

\def\NP{\mathsf{NP}}
\def\RP{\mathsf{RP}}
\def\bnsel{\mathsf{BN\mbox{-}Sel}}
\def\optsel{\mathsf{Opt\mbox{-}Sel}}
\def\NZ{\mathsf{NZ}}
\def\sparsity{\mathrm{sparsity}}
\def\scA{{\mathcal A}}
\def\scG{{\mathcal G}}
\def\EX{\mathsf{EX}}
\def\E{\mathbb{E}}
\def\moo{{\{-1, 1\}}}
\def\reals{\mathbb{R}}
\def\one{\mathds{1}}
\def\OS{\mathsf{OS}}
\def\OU{\mathsf{OU}}
\def\PAC{\mathsf{PAC}}
\def\junta{{\mathcal J}}
\def\Poisson{\mathrm{Poisson}}
\def\yy{{\mathcal Y}}
\def\poly{\mathrm{poly}}
\def\Nu{{\mathcal V}}
\def\transpose{T}
\def\th{{^{\textit{th}}}}
\def\lerr{\mathrm{\ell_2\mbox{-}err}}
\def\wmin{w_{\mbox{\footnotesize{min}}}}
\def\wmax{w_{\mbox{\footnotesize{max}}}}
\def\corr{\mathrm{corr}}
\def\argmin{\mathrm{argmin}}
\def\var{\mathrm{var}}
\def\lerror{\mathrm{\ell_2\mbox{-}error}}
\def\sparseset{\mathrm{sparse\mbox{-}set}}
\def\Mut{\mathsf{Mut}}
\def\Neigh{\mathsf{Neigh}}
\def\naturals{\mathbb{N}}
\def\Bene{\mathsf{Bene}}
\def\Neut{\mathsf{Neut}}
\def\opt{\mathsf{opt}}
\def\best{\mathsf{Best}}
\def\evalg{{\mathcal EA}}
\def\Sel{\mathsf{Sel}}
\def\Dists{{\mathcal D}}
\def\loss{\mathrm{L}}
\def\lin{\mathsf{Lin}}

\newcommand{\dtv}[1]{{ \Vert #1 \Vert_{\mbox{\footnotesize{TV}}}}}
\newcommand{\pinorm}[1]{{ \Vert #1 \Vert_{\pi}}}
\newcommand{\lznorm}[1]{\mathrm{sparsity}(#1)}
\newcommand{\ltwonorm}[1]{\Vert #1 \Vert}
\newcommand{\ip}[2]{\langle #1, #2 \rangle}

\newtheorem{lemma}{Lemma}
\newtheorem{theorem}{Theorem}
\newtheorem{remark}{Remark}
\newtheorem{definition}{Definition}
\newtheorem{proposition}{Proposition}
\newtheorem{claim}{Claim}


\begin{document}
\title{Attribute-Efficient Evolvability of Linear Functions} 

\author{
Elaine Angelino \\ Harvard University \\ \texttt{elaine@eecs.harvard.edu} 
\and 
Varun Kanade \\ UC Berkeley \\ \texttt{vkanade@eecs.berkeley.edu}
}

\maketitle

Darwin's theory of evolution through natural selection has been a cornerstone of
biology for over a century and a half.  Natural selection acts on
\emph{phenotypes} that are a result of molecular activities at the level of
cells; the molecular activities themselves are encoded in the DNA sequence, or
\emph{genotype}.\footnote{There are epigenetic factors at play, but the general
principle expounded in this document applies to those mechanisms as well.}
Every cell is a computing machine, sensing its environment and producing
appropriate responses.  Abstractly, cellular activity can be viewed as computing
a function; the input is an ``internal representation'' of the environment using
proteins and other molecules, and the output is the activation or repression
of proteins expression. Yet, a quantitative theory of the complexity of such
functions that could arise through Darwinian mechanisms has remained virtually
unexplored. Here, complexity is viewed through the lens of theoretical
computer science, \eg the size of the circuit required to compute the
function (cf. \citeN{Arora-Barak:textbook},
\citeN{Papadimitriou:textbook}).

To address this question, \citeN{Valiant:2009-evolvability} introduced a
computational model of evolution.\footnote{Also, see
\citeN{Valiant:2013-PAC} for a very accessible treatment of computational
learning theory in general, and this model in particular.} In this model, an
organism is an entity that computes a function of its environment. For
simplicity, each organism computes only one function, though in reality there
may be thousands if not more. There is a (possibly hypothetical) \emph{ideal
function} indicating the best behavior in every possible environment. The
performance of the organism is measured by how close the function it computes is
to the ideal. An organism produces a set of offspring, that may have mutations
that  alter the function computed. The performance measure acting on a
population of mutants forms the basis of natural
selection.\footnote{Recombination may increase the speed of evolution, but it is
understood that in Valiant's model, it does not affect the complexity that can
arise in functions.} The resources allowed are the most generous while remaining
feasible; the mutation mechanism may be any efficient randomized Turing
machine,\footnote{A Turing machine is a mathematical model on which all modern
computers are based; the widely believed Church-Turing hypothesis states that
\emph{any} feasible computation in nature can be simulated by a Turing machine.}
and the function represented by the organism may be arbitrary as long as it is
computable by an efficient Turing machine.

Formulated this way, the question of evolvability can be asked in the language
of computational learning theory. For what classes of ideal functions, $C$, can
one expect to find an evolutionary mechanism that gets arbitrarily close to the
ideal, within feasible computational resources? A function class captures a
notion of complexity, say for example,
%%
\[ f(x_1, \ldots, x_n) = 2 x_1 + 3.7 x_4 - 7.8 x_9 + 6 x_{12} + 18, \]
%%
is a linear function, and one could consider the class of all such linear
functions. A more complex class is that of quadratic functions, where a function
takes the form $f(x_1, \ldots, x_n) = 3.2 x_1^2 - 7.3 x_1 x_6 + 18 x_3 + 8.1$.
The point here is that the highest degree of any term appearing the expression
is $2$. One expects that the more \emph{complex} the ideal function, the harder
it is for evolution to succeed in approximating it. Here, the notion of
approximation is the following: Suppose there is a distribution $D$ over inputs
$(x_1, \ldots, x_n)$, the ideal function is $f$ and the organism computes some
other function $h$. Then, the \emph{loss} of the organism is $\E_{x \sim
D}[(f(x) - h(x))^2]$, the expected squared difference between the
organism and the ideal.\footnote{The closer $h$ is to $f$,
the smaller the loss. Here, we are simply assuming that the
performance depends on the approximation in terms of squared error. It is possible
to use other loss functions and indeed a very interesting (and largely
unresolved) question is how robust these evolutionary mechanisms are to changes
in the loss function.} An evolutionary mechanism is successful if the loss
becomes \emph{close} to $0$ (which is the best possible) in a relatively small
number of generations.

The reason for allowing mutations and representations of functions to be quite
general in Valiant's model is primarily the lack of our current understanding of
how mutations occur in nature and also the relationship betewen changes in genotype to
changes in phenotype.\footnote{It also has the added advantage that this allows
one to talk about computationally feasible Darwinian evolution of any type, not
just restricted to that observed on earth.} However, a consequence of this
generality has been that \emph{feasible} evolutionary mechanisms in this model
can be encodings of very sophisticated algorithms. These computations
have to be performed by ``chemical computers'' in cells, which could be severely
restricted in their abilities to perform computation. Thus, we are interested in
understanding what evolutionary mechanisms could succeed when they have limited
computational power at their disposal. We illustrate our point with a particular
kind of biological circuit: transcription networks.

The view presented here is an exaggerated simplification of the actual
transcription process, the goal being to focus on the complexity of the circuit
representation. A transcription network consists of 
interacting genes and proteins that are involved in the production of new
protein. Genes are \emph{transcribed} to produce mRNA, which is then
\emph{translated} into sequences of amino acids that ultimately fold into
proteins.\footnote{In reality, this is a dynamical system where the rates of
production are important. Note that this process need not be linear: a gene (mRNA
transcript) can be transcribed (translated) multiple times, not only in series
but also in parallel fashion.}
In a transcription network, a gene's transcription may be regulated by a set of
proteins called \emph{transcription factors}.
These factors may increase or decrease a gene's transcription by
physically binding to regions of DNA that are typically close to the gene.
In natural systems, only a small number of transcription factors
regulate any single gene, and so transcription networks are sparsely connected.
%%For example, Balaji~\etal studied a yeast
%%transcription network of 157 transcription factors regulating 4,410 genes. They
%%observed this network to have 12,873 interactions (edges) where each gene was
%%regulated on average by about 2.9 transcription factors, the distribution of
%%in-degrees was well-described by an exponential fit, and only about 45 genes had
%%an in-degree of 15 or greater~\cite{Balaji:2006}.

The number of transcription factors varies from hundreds in a bacterium to
thousands in a human cell. Some transcription factors are always present in the
cell and can be thought of as representing a \emph{snapshot} of the environment
(cf. \citeN{Alon:2006}). For example, the presence of sugar molecules in the
environment may cause specific transcription factors to be \emph{activated},
enabling them to regulate the production of other proteins.  One of these
proteins could be an \emph{end-product}, such as an enzyme that catalyzes a
metabolic reaction involving the sugar. Alternatively, the transcription factor
could regulate another transcription factor that itself regulates other genes --
we view this as intermediate computation -- and may participate in further
``computation'' to produce the desired end-result.
%
While transcription networks may include cycles (loops), here for simplicity we
focus on systems that are directed acyclic graphs, and the resulting computation
can be viewed as a circuit. 
%We illustrate a small, real transcription network in Figure~\ref{fig:network}. 
These circuits are by necessity shallow due to a temporal constraint, that the
time required for sufficient quantities of protein to be produced is of the same
order of magnitude as cell-division time.\footnote{Other kinds of networks, such
as signaling networks, operate by changing the shapes of proteins. The fact that
these transformations are rapid may allow for much larger depth. Note that fast
conformational changes govern how transcription factors directly process
information from the environment in order to regulate gene expression.  In our
example, a sugar molecule binds to a transcription factor and changes its shape
in a way that alters its ability to bind to DNA.} 
%
%For example, Luscombe~\etal measured the shortest path length (in number of
%intermediate nodes) between transcription factors and regulated genes
%corresponding to terminal nodes (leaves) in a yeast transcription network. In
%the static network, the mean such path length was 4.7 and the longest path
%involved 12 intermediate transcription factors~\cite{Luscombe:2004}.

In our work, we look at linear functions, \eg $f(x) = 3x_1 + 7 x_3 - 5 x_8$.  A
model to compute linear functions is an arithmetic circuit with one addition
gate and several input wires. Each input wire can have a positive or negative weight
associated with it.  Thus, the linear function $f(x) = 3x_1 + 7 x_3 -5x_8$, can
be computed by a simple circuit that has one addition gate and three input wires
having weight $3$, $7$ and $-5$ respectively. We now want to know that if the
\emph{ideal function} is sparse, \ie most of the weights are zero, does there
exist an evolutionary mechanism that would be successful, while having the
property that the ``cellular circuit'' at each stage of evolution is itself a
sparse one? Under rather mild assumptions on the distribution of inputs, we show
that this is indeed the case. A further interesting property of this mechanism
is that the main resource required, the number of generations, depends only on
the number of relevant variables in the \emph{ideal function}, and not on the
total number of variables.

Of course, a linear function may be a very restrictive model of what actually
goes on in the cell. As part of future work, we think it is worth investigating
whether other functions such as the sigmoid operating on linear functions can also
be evolved, assuming that the number of relevant variables is much smaller than
the total number of variables.

\subsection*{Bibliographic Note}

From the point of view of computer science, there have been very interesting
developments in understanding the power of Valiant's model. We have largely left
these out of this document for reasons of brevity and accessibility. The
interested reader is referred in particular to the work of Vitaly Feldman and
Paul Valiant. The second author's thesis contains an exposition of some of those
results and other work.

In the case of biological literature, our omissions are even greater. The
companion paper to this note~\cite{AK:2013} contains a fuller bibliography. Our
main purpose here is to introduce the computational framework in a language that
is more widely understood.

\subsection*{Acknowledgments} 
We would like to thank Leslie Valiant for helpful discussions and comments on an
early version of this paper. We are grateful to Frank Solomon for discussing
biological aspects related to this work. VK was supported by a Simons
Postdoctoral Fellowship.

\bibliography{all-refs}
\bibliographystyle{acmtrans}

\end{document}
