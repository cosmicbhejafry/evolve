We define a mutator that, under strong enough selection,
minimizes $\lerr_D(w, w^*)$ in a greedy fashion with respect to $\sparseset(w)$.
For $w \in R$, define the mutated representation $w^\prime$, output by the mutator as:
\begin{enumerate}
\item With probability $1/2$ do nothing, set $w^\prime = w$.
\item (Scaling) With probability $1/4 - \lambda/2$, choose $\gamma \in [-1, 1]$
uniformly at random and let $w^\prime = \gamma w$.
\item (Adjusting) With probability $1/4 - \lambda/2$, do the following.
Let $S_w = \{ i~|~ w_i \neq 0\}$. Pick $i \in S_w$ uniformly at random.
Let $w^\prime$ denote the mutated representation, where
$w^\prime_j = w_j$ for $j \neq i$, $w^\prime_i \in [-B, B]$ uniformly at random.
\item (Adding) With the remaining probability $\lambda$, do the following.
If $|S_w| < K$, choose $i \in [n] \setminus S_w$ uniformly at random.
Let $w^\prime$ be the mutated representation, where $w_j^\prime = w_j$ for
$j \neq i$, $w^\prime_i$ is chosen uniformly in $[-B, B]$.
Otherwise, if $|S_w| = K$, do nothing, set $w^\prime = w$.
\end{enumerate}

We present a selection framework that results in greedy progress in the sense
that, with high probability, $\lerr_D(w, w^*_S)$ is minimized for fixed $|S|$
before an adding step increments $|S|$.
Specifically, selection initializes $w = 0$ so that $\sparseset(w) = \emptyset$,
then greedily identifies $\sparseset(w^*)$ such that
$\sparseset(w) \subseteq \sparseset(w^\prime) \subseteq \sparseset(w^*)$
throughout the selection process.
This is inspired by and can be thought of as a robust version of a
greedy algorithm known in the signal processing literature as
orthogonal matching pursuit (OMP)~\cite{donoho48,tropp156}.
For an excellent review of related algorithms and issues from the perspective
of compressed sensing, see~\cite{donoho}.

\begin{enumerate}
\item (Initialization) Selection initializes $w = 0$, so $\sparseset(w) = \emptyset$.
\item (Main iteration)
	\begin{enumerate}
	\item Choose one class of mutations at random according to their probabilities.
	\item Generate $M > 2B(n - K) / \delta$ (+ extra?) mutants uniformly at random within that class.
	\item Set $w^\prime$ to the winner.
	\end{enumerate}
\end{enumerate}

In the ensuing discussion, we will prove that for any target function
$w^* \in C^k_{l,u}$, if $w \in R$ is such that
$\lerr_D(w, w^*) = \ltwonorm{w - w^*}^2 > \epsilon$,
then the mutator outputs a mutation which decreases the error by a
non-negligible amount with non-negligible probability.
We will establish our claim by proving the following claims.

\begin{enumerate}
\item[Claim A] For $\lambda$ small enough, if $\ltwonorm{w - w^*_S}^2 > \epsilon$,
then with non-negligible probability, the mutator will output a mutation $w^\prime$ such
that $\sparseset(w^\prime) = \sparseset(w)$ and it reduces $\lerror$ by at least some amount.
\item[Claim B] If $\ltwonorm{w - w^*_S}^2 < \epsilon$, then $w^\prime$ will be
such that either $\sparseset(w^\prime) = \sparseset(w)$ and
$\ltwonorm{w^\prime - w^*_S}^2 < \epsilon$, or
$\sparseset(w^\prime) \setminus \sparseset(w) = \{i\}$ where $i \in \sparseset(w^*)$
and $w^\prime$ reduces $\lerror$ by at least some amount.
\end{enumerate}

To establish Claim A above,

For Claim B, in the case that selection chose the adding class of mutations,
we must show that the new index $i \in \sparseset(w^*)$, where
$\{i\} = \sparseset(w^\prime) \setminus \sparseset(w)$.
We show that among the set of mutants output by the mutator, with high probability
there will exist at least one mutant such that it reduces $\lerror$ by something.

Let $\hat{w}$ be the best approximation of $w^*$ such that $\hat{w}_i \in [-B, B]$
for some $i \in [n] \setminus \sparseset(w)$ and $\hat{w}_j = w_j$ for $j \neq i$;
in other words, $\hat{w}$ is the best possible output achieved by mutating $w$ with
one adding step.
In expectation, if $\lznorm{w} = k$, then $M / (n - k)$ of the generated mutants
$v$ will have $\sparseset(v) = \sparseset(\hat{w})$, and each will satisfy
$\ltwonorm{v - \hat{w}}^2 < \delta$ with probability $\delta/2B$.
Thus in expectation, $\delta M / 2B(n - k) > 1$ mutants will be
``$\delta$-close'' to $\hat{w}$.
The remaining mutants with $\sparseset(v) \neq \sparseset(\hat{w})$ have
$\ltwonorm{v - \hat{w}}^2 > \rho$ because the things aren't very correlated \dots

We must show that $i \in \sparseset(w^*)$ and furthermore that
$\ltwonorm{\hat{w} - w^*}^2 < \ltwonorm{w - w^*}^2 - something$.
We start with the initial representation, $w=0$.
Without loss of generality, arrange $w^*$ so that $\sparseset(w^*) = [K]$,
in decreasing order of the values \dots. Initially, the residual is $w^*$.
