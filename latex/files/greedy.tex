In this section, we present a different evolutionary mechanism for evolving
sparse linear functions. This algorithm essentially is an adaptation of a greedy
algorithm commonly known as orthogonal matching pursuit (OMP) in the signal
processing literature (see ~\cite{Donoho:2006-recovery, Tropp:2004-greed}). Our
analysis requires stronger properties on the distribution: we show that
$k$-sparse linear functions can be evolved with respect to $1/(2k)$-incoherent
$\Delta, G$-nice distributions (see Defn.~\ref{defn:bhutan}). Also, here the
selection rule used is selection using \emph{optimisation}
($\optsel$).\footnote{Valiant showed that selection using optimisation was
equivalent to selection using beneficial and neutral
mutations~\cite{Valiant:2009-evolvability}. However, this reduction uses
representation classes that may be somewhat complex. For restricted
representation classes, it is not clear that such a reduction holds. In
particular, the necessary ingredients seems to be \emph{polynomial-size}
memory.} And, the algorithm is guaranteed to succeed only with
\emph{initialization} from the $0$ function.  Nevertheless, this evolutionary
algorithm is appealing due to its simplicity and also because it is a
\emph{proper} evolutionary mechanism -- it never uses a representation that is
not a $k$-sparse linear function.

Recall that $f \in  \lin^k_{0, u}$ is the ideal (target)
function.\footnote{Here, we no longer need the fact that each coefficient in the
target linear function has magnitude at least $l$.} Let 
%
\[ R = \{w ~|~ \sparsity(w) \leq k, w_i \in [-B, B] \}, \]
%
where $B = 10uk/\Delta$. Now, starting from $w \in R$, define the action of the
mutator as follows (we will define the parameters $\lambda$ and $m$ later in the
proof of Theorem~\ref{thm:greedy}):
%%
\begin{enumerate}
%%
\item {\em Adding}: With probability $\lambda$ do the following: Recall $\NZ(w)$
denotes the non-zero entries of $w$. If $|\NZ(w)| < k$, choose $i \in [n]
\setminus \NZ(w)$ uniformly at random. Let $w^\prime$ be such that $w^\prime_j =
w_i$ for $j \neq i$ and $w^\prime_i \in [-B, B]$ uniformly at random. If
$\NZ(w)= k$, let $w^\prime = w$. Then, the multiset $\Neigh(w, \epsilon)$ is
populated by $m$ independent draws from the procedure just described.
%
\item With probability $1 - \lambda$ do the following:
%
\begin{enumerate}
\item {\em Identical}: With probability $1/2$ output $w^\prime = w$
\item {\em Scaling}: With probability $1/4$, choose $\gamma \in [-1, 1]$ uniformly at
random and let $w^\prime = \gamma w$.
%
\item {\em Adjusting}: With probability $1/4$, do the following.  Pick $i \in \NZ(w)$
uniformly at random.  Let $w^\prime$ be such that $w^\prime_j = w_j$ for $j \neq
i$, $w^\prime_i \in [-B, B]$ uniformly at random.
\end{enumerate}
Then, the multiset $\Neigh(w, \epsilon)$ is populated by $m$ independent draws
from the procedure just described.
\end{enumerate}

One thing to note in the above definition is that the mutations produced by the
mutator at any given time are correlated -- \ie they are all either of the kind
that add a new variable, or all of the kind that just manipulate existing
variables.  At a high level, we prove the success of this mechanism as follows:
\begin{enumerate}
%
\item Using mutations of type ``scaling'' or ``adjusting,'' a representation
that is close to the \emph{best} in the space, \ie $f^S$, is evolved.
%
\item When the representation is (close to) the best possible using current
variables, adding one of the variables that is present in the \emph{ideal
function}, but not in the current representation, results in the greatest
reduction of expected loss. Thus, selection based on optimisation would always
add a variable in $\NZ(f)$. By tuning $\lambda$ appropriately, it is ensured
that candidate mutations that add new variables are never chosen until evolution
has had time to approach the \emph{best} representation using existing
variables.
%
\end{enumerate}

To complete the proof we establish the following claims.

\begin{claim} \label{claim:date} If $\ltwonorm{f^S - w} \leq
\sqrt{\epsilon}/{2k}$, then if $S \subsetneq \NZ(f)$, there exists $i \in \NZ(f)
\setminus S$ and $-B < a < b < B$, such that for any $\gamma \in [a, b]$,
$\loss_{f, D}(w + \gamma e^i) \leq \loss_{f, D}(w) - \epsilon/(4k^2)$ and for
any $j \not\in \NZ(f)$, $\beta \in [-B, B]$, $\loss_{f, D}(w + \beta e^j) \geq
\loss_{f, D}(w + \gamma e^i) + \epsilon/(4k^3)$. Futhermore, $b - a \geq
\sqrt{(k+1) \epsilon}/k^2$. \end{claim}

\begin{claim} \label{claim:elderberry} Conditioned on the mutator not outputting
mutations that add a new variable, with probability at least $\min\{1/16,
\ltwonorm{f^S - w}/(16k^2)\}$, there exists a mutation that reduces the squared
loss by at least $\ltwonorm{f^S - w}^2/(12k^2)$. \end{claim}

The proofs of Claims~\ref{claim:date} and \ref{claim:elderberry} are not
difficult and are provided in Appendix~\ref{app:greedy}. Based on the above
claims we can prove the following theorem:

\begin{theorem} \label{thm:greedy} Let $\Dists$ be the class of $\Delta$-smooth
$G$-nice distributions over $\reals^n$ (Definition~\ref{defn:afghanistan}).
Then, the class $\lin^k_{0, u}$ is evolvable by an evolutionary algorithm, using
the mutation algorithm described in this section, selection rule $\optsel$, and
the representation class $R = \lin^k_{0, B}$, where $B = 10 uk/\Delta$.
Furthermore, the following are true:
%%
\begin{enumerate}
\item The number of generations $g$ is independent of the dimension $n$.
\item The size of the sample required for estimating empirical losses depends
only polylogarithmically on $n$.
\end{enumerate}
%%
\end{theorem}
\begin{proof} The proof is straightforward, although heavy on notation;
we provide a sketch here. The mutator is as described in this section. Let
%%
\[ p = \min\{1 /16, \sqrt{\epsilon}/(16k^2 B), \sqrt{(k+1)\epsilon}/k\}, \]
%%
and let
%%
\[ \alpha = \min\{\epsilon/(4k^3), \epsilon/(192k^2)\}. \]
%%
Also let $\tau = \alpha/5$ and $t = 3\alpha/5$.

Consider a sequence of coin tosses, where the probability of heads is $\lambda$
and the probability of tails is $1 - \lambda$. Let $Y_i$ be the number of tails
between the $(i-1)\th$ and $i\th$ heads. Except with probability
$\epsilon/(4(k+1))$, $Y_i > \epsilon/(4(k+1)\lambda)$ by a simple union bound.
Also, by Markov's inequality, except with probability $\epsilon/(4(k+1))$, $Y_i
< 4(k+1)/(\epsilon\lambda)$. Thus, except with probability $\epsilon/2$, we have
$\epsilon/(4(k+1)\lambda) \leq Y_i \leq 4(k+1)/(\epsilon\lambda)$ for $i = 1, 2,
\ldots, k+1$. 

The heads in the above experiment correspond to time steps when the mutator
produces mutations of type ``adding,'' and the tails correspond to
all the remaining mutations.
Thus, let $g = 4(k+1)^2/(\epsilon\lambda) + k + 1$. Except with probability
$\epsilon/2$, it must be the case that there have been at least $k + 1$
heads, and furthermore, between the $(i-1)\th$ and $i\th$ heads there have
been at least $\epsilon/(4(k+1)\lambda)$ tails. We will show that the steps with
tails are sufficient to ensure that evolution reaches the (almost) best
possible target in the variables available to it. In particular, if the set of
available variables is $S$, the representation $w$ reached by evolution will
satisfy $\ltwonorm{f^S - w}^2 \leq \epsilon/(2k^2)$. For now, suppose that this
is the case. 

Also, let $m = p^{-1}\ln(4g/\epsilon)$ and $s = (800 kB^2G^2/\alpha^2)
\ln(4mg/\epsilon)$. These values ensure that for $g$ generations, except with
probability $\epsilon/2$, the mutator always produces a mutation that had
probability at least $p$ of being produced (conditioned on ``heads'' or
``tails''), and that for all the representations concerned, $|\hat{\loss}_{f,
D}(w) - \loss_{f, D}(w)| \leq \tau$, where $\tau = \alpha/5$. Thus, allowing the
process to fail with probability $\epsilon$, we assume that none of the
undesirable events have occurred.

We claim by induction that evolution never adds a ``wrong'' variable, \ie one
that is not present in the target function $f$. The base case is trivially true,
since the starting representation is $0$. Now suppose, just before a ``heads''
step, the representation is $w$, such that $S = \NZ(w)$ and $\ltwonorm{f^S - w}
\leq \epsilon/(2k^2)$. The current step is assumed to be a ``heads'' step, thus
the mutator has produced mutations by adding a new variable. Then, using
Claim~\ref{claim:date}, we know that there is a mutation $w^\prime$ in $\Neigh(w,
\epsilon)$, such that $\loss_{f, D}(w^\prime) < \loss_{f, D}(w) -
\epsilon/(4k^2)$ (obtained by adding a correct variable). For the settings of
tolerance $t$ and $\tau$, this ensures that the set $\best$ for selection by
optimisation is non-empty. Also, by the latter half of Claim~\ref{claim:date},
it is ensured that no mutation that adds an \emph{irrelevant} variable can
appear in the set $\best$. Thus, evolution will choose a mutation that adds a
\emph{relevant} variable.

Finally, note that during a ``tails'' step, as long as $\ltwonorm{f^S - w}^2
\geq \epsilon/(4k^2)$, there exists a mutation that reduces the expected loss by
at least $\epsilon^2/(192k^2)$. This implies that the set $\best$ is non-empty
and for the values of tolerance $t = 3\alpha/5$ and $\tau = \alpha/5$, any
mutation from the set $\best$ reduces the expected loss by at least $\alpha/5$.
Since the maximum loss is at most $4kB^2G^2$ for the class of distributions and
a representation $w$ from the set $R$, in at most $20kB^2G^2/\alpha$ steps, a
representation satisfying $\loss_{f, D}(w) \leq \epsilon/(4k^2)$ must be
reached. Note that subsequently, it is ensured that the loss does not increase
substantially, since with probability at least $(1 - \lambda)/2$, the mutator
outputs the same representation. Hence, it is guaranteed that there is always a
neutral mutation. Thus, if $\lambda$ is set to $\epsilon\alpha/(80
k(k+1)B^2G^2$, the evolutionary algorithm using the selection rule $\optsel$
succeeds.

\end{proof}


%% VK %% In the ensuing discussion, we will prove that for any target function
%% VK %% $w^* \in C^k_{l,u}$, if $w \in R$ is such that
%% VK %% $\lerr_D(w, w^*) = \ltwonorm{w - w^*}^2 > \epsilon$,
%% VK %% then the mutator outputs a mutation which decreases the error by a
%% VK %% non-negligible amount with non-negligible probability.
%% VK %% We will establish our claim by proving the following claims.
%% VK %% 
%% VK %% \begin{enumerate}
%% VK %% \item[Claim A] If $\ltwonorm{w} \ge 2 \ltwonorm{w^*_S}$, then for $M \ge 1$,
%% VK %% with probability at least $1/16 - \lambda/8 \ge $ \eanote{something} the mutator outputs
%% VK %% a mutation that reduces $\lerror$ by at least $\ltwonorm{w - w^*_S}^2 / 12$.
%% VK %% \item[Claim B] If $\ltwonorm{w} \le 2 \ltwonorm{w^*_S}$ and
%% VK %% $\ltwonorm{w - w^*_S} \ge \alpha$, then with probability at least $\alpha$,
%% VK %% there exists a mutation that decreases the $\lerror$ by at least $\alpha$.
%% VK %% \item[Claim C] If $\Delta^2 > 1 - 1/(2 K - 1)$ and
%% VK %% $\ltwonorm{w - w^*_S}^2 < \epsilon$, then $w^\prime$ will be
%% VK %% such that either $\sparseset(w^\prime) = \sparseset(w)$ and
%% VK %% $\ltwonorm{w^\prime - w^*_S}^2 < \epsilon$, or
%% VK %% $\sparseset(w^\prime) \setminus \sparseset(w) = \{i\}$ where $i \in \sparseset(w^*)$
%% VK %% and $w^\prime$ reduces $\lerror$ \eanote{by at least some amount}.
%% VK %% \end{enumerate}
%% VK %% 
%% VK %% To establish Claims A and B above, we follow the lines of reasoning from the
%% VK %% previous section.
%% VK %% For Claim A, we consider the case where selection has chosen the scaling class
%% VK %% of mutants.  With the assumption that $\ltwonorm{w} \ge 2 \ltwonorm{w_S^2}$ and
%% VK %% for each of the generated mutants $\gamma w$ with $\gamma \in [1/2, 3/4]$,
%% VK %% we obtain
%% VK %% 
%% VK %% \[
%% VK %% \ltwonorm{\gamma w - w^*_S}^2 \le \ltwonorm{w - w^*_S}^2 - \frac{1}{12}\ltwonorm{w - w^*_S}^2.
%% VK %% \]
%% VK %% 
%% VK %% For Claim B, \dots
%% VK %% 
%% VK %% For Claim C, in the case that selection chose the adding class of mutations,
%% VK %% we must show that the new index $i \in \sparseset(w^*)$, where
%% VK %% $\{i\} = \sparseset(w^\prime) \setminus \sparseset(w)$.
%% VK %% We show that among the set of mutants output by the mutator, with high probability
%% VK %% there will exist at least one mutant such that it reduces $\lerror$ by something.
%% VK %% \eanote{Explain how this relates to the proof in Donoho and give the correct citation(s)}.
%% VK %% 
%% VK %% Let $\hat{w}$ be the best approximation of $w^*$ such that $\hat{w}_i \in [-B, B]$
%% VK %% for some $i \in [n] \setminus \sparseset(w)$ and $\hat{w}_j = w_j$ for $j \neq i$;
%% VK %% in other words, $\hat{w}$ is the best possible output achieved by mutating $w$ with
%% VK %% one adding step.
%% VK %% In expectation, if $\lznorm{w} = k$, then $M / (n - k)$ of the generated mutants
%% VK %% $v$ will have $\sparseset(v) = \sparseset(\hat{w})$, and each will satisfy
%% VK %% $|v_i - \hat{w}_i| \le \delta$, $\delta < B$, and so
%% VK %% $\ltwonorm{v - \hat{w}} \le \delta$ with probability $\delta/2B$.
%% VK %% Thus in expectation, for $\delta > 2B(n-k) / M$, at least one mutant will be
%% VK %% ``$\delta$-close'' to $\hat{w}$.
%% VK %% Below we argue that the remaining mutants with
%% VK %% $\sparseset(v) \neq \sparseset(\hat{w})$ will not be better than this mutant by
%% VK %% a margin of at least \eanote{fill in this condition}.
%% VK %% 
%% VK %% We must show that $i \in \sparseset(w^*)$ and furthermore that
%% VK %% \eanote{fill in this condition}.
%% VK %% Selection starts with the initial representation, $w=0$.
%% VK %% Without loss of generality, arrange $w^*$ so that $\sparseset(w^*) = [K]$,
%% VK %% in decreasing order of the values $\vert w^*_k \vert \sqrt{\var(X_k)}$.
%% VK %% Initially, the residual $w^* - w$ is $w^*$.
%% VK %% Let $w^*_{\{j\}}$ be the best approximation of $w^*$ with
%% VK %% $\sparseset(w^*_{\{j\}}) = \{j\}$, so $w^*_{\{j\}}$ \dots
%% VK %% With high probability, the mutator generates at least one mutant
%% VK %% $\hat{w}_{\{j\}}$ for each $j \in [n]$ that minimizes the residual to within
%% VK %% $\delta$ of the optimal solution restricted to $\{j\}$, i.e.,
%% VK %% $\ltwonorm{\hat{w}_{\{j\}} - w^*}^2 < \ltwonorm{w^*_{\{j\}} - w^*}^2 + \delta$.
%% VK %% The $\lerror$ of $w^*_{\{j\}}$ is
%% VK %% \eanote{Check the last step below!}
%% VK %% 
%% VK %% \begin{align*}
%% VK %% \ltwonorm{w^*_{\{j\}} - w^*}^2 = \min_{w_{\{j\}}}{\ltwonorm{w_{\{j\}} - w^*}}^2
%% VK %% &= \bigg \Vert e_j \frac{\ip{e_j}{w^*}}{\ltwonorm{e_j}^2} - w^* \bigg \Vert^2 \\
%% VK %% &= \ltwonorm{w^*}^2 - 2 \bigg \langle e_j \frac{\ip{e_j}{w^*}}{\ltwonorm{e_j}^2}, w^* \bigg \rangle + \bigg \Vert e_j \frac{\ip{e_j}{w^*}}{\ltwonorm{e_j}^2} \bigg \Vert^2 \\
%% VK %% &= \ltwonorm{w^*}^2 - \frac{(\ip{e_j}{w^*})^2}{\ltwonorm{e_j}^2} \ge 0.
%% VK %% \end{align*}
%% VK %% 
%% VK %% \noindent Note that
%% VK %% 
%% VK %% \[
%% VK %% \frac{(\ip{e_j}{w^*})^2}{\ltwonorm{e_j}^2}
%% VK %% = \frac{(\E[(e_j \cdot X)(w^* \cdot X)])^2}{\E[(e_j \cdot X)(e_j \cdot X)]}
%% VK %% = \frac{(\E[X_j (w^* \cdot X)])^2}{\E[X_j^2]}
%% VK %% = \frac{(\sum_{k=1}^K w^*_k \E[X_j X_k]))^2}{\var(X_j)}
%% VK %% \]
%% VK %% 
%% VK %% For selection to choose a mutant $\hat{w}_{\{j\}}$ with
%% VK %% $j \in \sparseset(w^*) = [K]$, it must be the case that for any $i > K$,
%% VK %% 
%% VK %% \[
%% VK %% \bigg\vert \frac{\ip{e_1}{w^*}}{\ltwonorm{e_1}} \bigg\vert
%% VK %% > \bigg\vert \frac{\ip{e_i}{w^*}}{\ltwonorm{e_i}} \bigg\vert + \eanote{~something}.
%% VK %% \]
%% VK %% 
%% VK %% We construct lower and upper bounds for the left- and right-hand sides,
%% VK %% respectively, making use of the fact that
%% VK %% $|\corr(X_i, X_j)| \leq 1 - \Delta^2$ from Lemma~\ref{} and also
%% VK %% that we arranged the $\vert w^*_k \vert \sqrt{\var(X_k)}$ in decreasing order.
%% VK %% On the left,
%% VK %% 
%% VK %% \begin{align*}
%% VK %% \bigg\vert \frac{\ip{e_1}{w^*}}{\ltwonorm{e_1}} \bigg\vert
%% VK %% &= \frac{\vert\sum_{k=1}^K w^*_k \E[X_1 X_k]\vert}{\sqrt{\var(X_1)}} \\
%% VK %% &\ge \frac{\vert w^*_1 \E[X_1 X_1]\vert}{\sqrt{\var(X_1)}} - \frac{\sum_{k=2}^K \vert w^*_k \E[X_1 X_k] \vert}{\sqrt{\var(X_1)}} \\
%% VK %% &= \vert w^*_1 \vert \sqrt{\var(X_1)} - \sum_{k=2}^K \vert w^*_k \vert \sqrt{\var(X_k)}~\vert \corr(X_1, X_k) \vert \\
%% VK %% &\ge \vert w^*_1 \vert \sqrt{\var(X_1)}(1 - (1 - \Delta^2) (K-1))
%% VK %% \end{align*}
%% VK %% 
%% VK %% \noindent On the right,
%% VK %% 
%% VK %% \begin{align*}
%% VK %% \bigg\vert \frac{\ip{e_i}{w^*}}{\ltwonorm{e_i}} \bigg\vert
%% VK %% = \frac{\vert\sum_{k=1}^K w^*_k \E[X_i X_k]\vert}{\sqrt{\var(X_i)}}
%% VK %% \le \sum_{k=1}^K \vert w^*_k \vert \frac{\vert \E[X_i X_k]\vert}{\sqrt{\var(X_i)}}
%% VK %% &= \sum_{k=1}^K \vert w^*_k \vert \sqrt{\var(X_k)} \corr(X_i, X_k) \\
%% VK %% &\le \vert w^*_1 \vert \sqrt{\var(X_1)} (1 - \Delta^2) K.
%% VK %% \end{align*}
%% VK %% 
%% VK %% \noindent Putting these bounds together now gives the requirement
%% VK %% 
%% VK %% \[
%% VK %% \bigg\vert \frac{\ip{e_1}{w^*}}{\ltwonorm{e_1}} \bigg\vert
%% VK %% \ge \vert w^*_1 \vert \sqrt{\var(X_1)}(1 - (1 - \Delta^2) (K-1))
%% VK %% > \vert w^*_1 \vert \sqrt{\var(X_1)} (1 - \Delta^2) K
%% VK %% \ge \bigg\vert \frac{\ip{e_i}{w^*}}{\ltwonorm{e_i}} \bigg\vert,
%% VK %% \]
%% VK %% 
%% VK %% \noindent and so
%% VK %% 
%% VK %% \begin{align*}
%% VK %% 1 - (1 - \Delta^2) (K-1) = 2 - K + \Delta^2 K - \Delta^2 &> (1 - \Delta^2) K \\
%% VK %% %\Rightarrow & \qquad 1 + (1 - \Delta^2) > 2 (1 - \Delta^2) K
%% VK %% %\qquad \Rightarrow \qquad K < \frac{1}{2} \biggl( 1 + \frac{1}{1 - \Delta^2} \biggr)
%% VK %% \Rightarrow \qquad \Delta^2 &> 1 - \frac{1}{2 K - 1}
%% VK %% \end{align*}
%% VK %% 
%% VK %% \noindent  This is the condition for $\Delta^2$ in Claim C.
%% VK %% \eanote{Finally, argue that subsequent steps work too.}
%% VK %% \eanote{And then we should say something about the CS analog.}
