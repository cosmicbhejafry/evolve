We present a simple algorithm that evolves the class of sparse linear functions,
$\lin^k_{l, u}$.  The representation class also consists of sparse linear
functions, but with a greater number of non-zero entries than the \emph{ideal
function}. We also assume that a linear function is represented by $w \in
\reals^n$, where each $w_i$ is a real number. (Handling the issues of finite
precision is standard and is avoided in favour of simplicity.) Define the
parameters $K = 32(k^2/\Delta^4)(u/l)$ and $B = 10 uk /\Delta$. Formally, the
representation class is
\[ 
R = \{ w ~|~ \lznorm{w} \leq K, w_i \in [-B, B] \}
\]
The important point to note is that the parameters $K$ and $B$ do not depend on
$n$, the number of variables.

Next, we define the mutator. Recall that the mutator is a randomized algorithm
that takes as input an element $r \in R$, the accuracy parameter $\epsilon$,
and outputs a multiset $\Neigh(r, \epsilon) \subseteq R$. Here, $\Neigh(r,
\epsilon)$ is populated by $m$ independent draws from the following procedure,
where $m$ will be specified later. Starting from $w \in R$, define the mutated
representation $w^\prime$, output by the mutator as:
%%
\begin{enumerate}
%
\item (Scaling) With probability $1/3$, choose $\gamma \in [-1, 1]$ uniformly at
random and let $w^\prime = \gamma w$. 
%
\item (Adjusting) With probability $1/3$, do the following. Recall that $\NZ(w)
= \{ i~|~ w_i \neq 0 \}$. Pick $i \in \NZ(w)$ uniformly at random. Let
$w^\prime$ denote the mutated representation, where $w^\prime_j = w_j$ for $j
\neq i$, $w^\prime_i \in [-B, B]$ uniformly at random. 
%
\item With the remaining $1/3$ probability, do the following:
\begin{enumerate}
\item (Swapping) If $|\NZ(w)| = K$, choose $i_1 \in \NZ(w)$ uniformly at random.
Then, choose $i_2 \in [n] \setminus \NZ(w)$ uniformly at random. Let $w^\prime$
be the mutated representation, where $w_j^\prime = w_j$ for $j \neq i_1, i_2$.
Set $w_{i_1}^\prime = 0$ and choose $w_{i_2}^\prime$ uniformly from the range
$[-B, B]$. Observe that in this case, $\sparsity(w^\prime) = \sparsity(w) = K$
with probability $1$, and that $w^\prime \in R$.
\item (Adding) If $|\NZ(w)| < K$, choose $i \in [n] \setminus \NZ(w)$ uniformly
at random. Let $w^\prime$ be the mutated representation, where $w_j^\prime =
w_j$ for $j \neq i$, $w^\prime_i$ is chosen uniformly in $[-B, B]$. 
\end{enumerate}
\end{enumerate}

In the ensuing discussion, we fix the following notation: $f \in C^k_{l, u}$ is
the target linear function, $D$ is the underlying distribution and is a
$\Delta$-smooth bounded distribution (see Section~\ref{sec:notation-class}). Recall
that for linear functions, $w, w^\prime$, we have the inner product
$\ip{w}{w^\prime} = \E_{x \sim D}[(w \cdot x) (w^\prime \cdot x)]$, which also
defines the norm $\ltwonorm{w} = \sqrt{\ip{w}{w}}$.  We will show that for any
$w \in R$, if $\loss_{f, D}(w) = \ltwonorm{f - w}^2 > \epsilon$, with non-negligible
(inverse polynomial) probability, the above procedure produces a mutation
$w^\prime$, such that $\loss_{f, D}(w^\prime) = \ltwonorm{f - w^\prime} \leq
\loss_{f, D}(w) - b$, for some inverse polynomial $b$.

To simplify notation, let $S = \NZ(w)$. Let $f^S$ denote best approximation to
$f$ using variables in the set $S$. Formally, 
%
\[ 
f^S = \underset{w \in R ~:~ w_i = 0 ~\vee~ i \in S} \argmin \ltwonorm{f- w}^2 
\]

Let $w^\prime$ denote a random mutation produced as a result of the procedure
described above.  We will establish the desired result by proving the following
claims.
%%
\begin{claim} \label{claim:apple} If $\ltwonorm{w} \geq 2 \ltwonorm{f^S}$, then
with probability at least $1/12$, $\loss_{f, D}(w^\prime) \leq \loss_{f, D}(w) -
\ltwonorm{w - f^S}^2/12$. In particular, a ``scaling'' type mutation achieves
this. \end{claim}
%%
\begin{claim} \label{claim:banana} When $\ltwonorm{w} \leq 2 \ltwonorm{f^S}$,
then with probability at least $\Delta \ltwonorm{f^S - w}/(6K^2 B)$, there
exists a mutation that decreases the expected loss by at least $3
\Delta^2\ltwonorm{f^S - w}^2/(4|S|^2)$. In particular, an ``adjusting'' type
mutation achieves this. \end{claim}
%%
\begin{claim} \label{claim:cantaloupe} When $\ltwonorm{f^S - w} \leq
l^2\Delta^2/(2KB)$, but $\sparseset(f) \not\subseteq S$, then with probability
at least $\Delta \ltwonorm{f - w}/(6KBnk)$, there exists a mutation that
decreases the expected loss by at least $\Delta^2 \ltwonorm{f-w}^2/(16k^2)$.
\end{claim}

Finally, note that when $\NZ(f) \subseteq S$, then $f^S = f$. Thus, in this case
when $\loss_{f, D}(w) = \ltwonorm{f^S - w}^2 \leq \epsilon$, the evolutionary
algorithm has succeeded. \medskip 

\vknote{I think we'll eventually structure this by saying -- the proofs of the
claim are in Appendix Hugh. Then, we'll have a short proof showing that if the
Claims are true, evolution indeed succeeds.}

\begin{proof}[Proof of Claim \ref{claim:apple}]: We show that in this case, a ``scaling''
mutation achieves the desired result. Restricted to the direction $w$, the best
approximation to $f^S$ is $\frac{\ip{w}{f^S}}{\ltwonorm{w}^2}w$. We have that
%%
\[
\left\Vert \frac{\ip{w}{f^S}}{\ltwonorm{w}^2} w \right\Vert \leq
\ltwonorm{f^S} \leq \frac{\ltwonorm{w}}{2}
\]
%%
Hence, if $\ip{w}{f^S} > 0$, for $\gamma \in [1/4, 3/4]$ (and similarly if
$\ip{w}{f^S} < 0$ for $\gamma \in [-3/4, -1/4]$ ), we have that
%%
\begin{align*}
\ltwonorm{\gamma w - f^S}^2 &= \ltwonorm{w- f^S}^2 - 2 (1 - \gamma) \ip{w -
f^S}{w} + (1 - \gamma)^2 \ltwonorm{w}^2 \\
&\leq \ltwonorm{w - f^S}^2 - (1 - \gamma) \ltwonorm{w}^2 + (1 - \gamma)^2
\ltwonorm{w}^2 \\
&= \ltwonorm{w - f^S}^2 - (\gamma - \gamma^2) \ltwonorm{w}^2
\intertext{Finally, by observing that for $\gamma \in [1/4, 3/4]$, $\gamma -
\gamma^2 \geq 3/16$ and that by the triangle inequality, $\ltwonorm{w} \geq
(2/3)\ltwonorm{w - f^S}$ when $\ltwonorm{w} \geq 2 \ltwonorm{f^S}$, we obtain}
%
\ltwonorm{\gamma w - f^S}^2 &\leq \ltwonorm{w - f^S}^2 - \frac{1}{12}
\ltwonorm{w - f^S}^2
\end{align*}
We note that $\loss_{f, D}(w^\prime) = \ltwonorm{f - f^S}^2 + \ltwonorm{f^S -
\gamma w}^2$ and $\loss_{f, D}(w) = \ltwonorm{f - f^S}^2 + \ltwonorm{f^S -
w}^2$.  An appropriate value of $\gamma$ is chosen with probability at least $1/4$,
and combined with the probability of choosing a scaling mutation we get the
desired result.
\end{proof}

\begin{proof}[Proof of Claim~\ref{claim:banana}] Here, we appeal to a mutation
that adjusts the relative weights of the variables within the set $S = \NZ(w)$.
Consider the vector $f^S - w$, and note that $\NZ(f^S -w ) \subseteq S$. Let
$r^S = f^S - w$ denote the residual, which lies in the space spanned by $S$.
Now consider
\begin{align*}
\ltwonorm{r_S}^2 = \ip{r^S}{r^S} &= \sum_{i \in S} r^S_i \ip{e^i}{r^S}
\intertext{Here, $e^i$ is the unit vector representing the linear function $x
\mapsto e^i \cdot x = x_i$. Therefore, there must exist an $i$ for which the
following is true:}
r^S_i\ip{e^i}{r^S} &\geq \frac{\ltwonorm{r^S}^2}{|S|}
\intertext{We appeal to Lemma~\ref{lemma:amsterdam} (part 1), which implies that
$|r^S_i| \leq \sqrt{\ip{r^S}{r^S}/\Delta^2} = \ltwonorm{r^S}/{\Delta}$, to
conclude that}
|\ip{e^i}{r^S}| &\geq \frac{\ltwonorm{r^S} \Delta}{|S|}
\end{align*}
Let $\beta = \frac{\ip{e^i}{r^S}}{\ltwonorm{e^i}^2}$ and suppose $w^\prime =
w + \gamma e^i$ for $\gamma \in [\beta - |\beta|/2, \beta +  |\beta|/2]$. We
then have
\begin{align*}
\ltwonorm{f^S - (w + \gamma e^i)}^2 &= \ltwonorm{w - f^S}^2 - 2 \gamma \ip{f^S -
w}{e^i} + \gamma^2 \ltwonorm{e^i}^2
%
\intertext{Recall that $f^S - w = r^S$ and note that $\ip{e^i}{r^S}$ and
$\gamma$ have the same sign. This, combined with the above equation, gives}
%
\ltwonorm{f^s - (w + \gamma e^i)}^2 &= \ltwonorm{w - f^S}^2 - (2
|\gamma||\beta| - \gamma^2) \ltwonorm{e^i}^2
\intertext{Finally, note that for $|\gamma| \in [|\beta|/2, 3|\beta|/2]$, $- 2
|\gamma| |\beta| + \gamma^2 \leq - 3 |\beta|^2/4$, hence, the
above equation and the fact that $\ltwonorm{e^i} \leq 1$ together yield}
\ltwonorm{f^S - (w + \gamma e^i)}^2 &\leq \ltwonorm{w - f^S}^2 - \frac{3}{4}
\beta^2 \ltwonorm{e^i}^2 \leq \ltwonorm{w - f^S}^2 - \frac{3}{4}
\frac{\ltwonorm{f^s - w}^2 \Delta^2}{|S|^2}
\end{align*}

Note that a suitable mutation $w^\prime = w + \gamma e^i$ is obtained with
probability at least $|\beta|/(6KB)$ ($1/3$ for choosing the right type of
mutation, $1/K$ for the correct choice of variable, and $|\beta|/(2B)$ for the
choosing the correct value of $w_i$).  Also note that
$|\beta| \ge \Delta \ltwonorm{f^S - w} / |S| \ge \Delta \ltwonorm{f^S - w} / K$.
For this to be a valid mutation, we also need to verify the fact that $w_i + \gamma
\in [-B, B]$, which is ensured by our choice of $B$. To see this, note that
$\ltwonorm{w} \leq 2 \ltwonorm{f^S} \leq 2 \ltwonorm{f}$ (the last part is
because $f^S$ is a projection of $f$ onto a lower dimensional space). Thus, by
Lemma~\ref{lemma:amsterdam} (part 1), $w_i \leq 2 \ltwonorm{f} / \Delta$. Also,
$|\beta| = |\ip{e^i}{r^S}|/\ltwonorm{e^i}^2 \leq \ltwonorm{r^S}/\ltwonorm{e^i}
\leq 3 \ltwonorm{f}/\Delta$. Thus, if $B > 10 \ltwonorm{f}/ \Delta$, the
mutation will be a valid one. Note that the maximum value of $\ltwonorm{f}$ for
$f \in \lin^k_{l, u}$ is $ku$. Thus, our choice of $B = 10 ku / \Delta$ is
sufficient to ensure that the required mutation is a valid one.  This completes
the proof of Claim~\ref{claim:banana}.
\end{proof}

\begin{proof}[Proof of Claim~\ref{claim:cantaloupe}] Finally, we show that if
$\ltwonorm{f^S - w}$ is very small, but $\NZ(f) \not\subseteq S$, then it must
be the case that a ``swapping'' or ``adding'' mutation is beneficial. We focus
on the swapping case, \ie when $|S| = K$; the adding step is a special case of
this. First, we observe that if there exists $i \in \NZ(f)$ such that $i \not\in
S$, then by using Lemma~\ref{lemma:amsterdam} (part 1), it must be the case that
$\ltwonorm{f - w}^2 \geq f_i^2 \Delta^2 \geq l^2 \Delta^2$. Let $r = f - w$
denote the residual. Then, consider the following:
%%
\begin{align}
%
\ip{r}{r} &=  \sum_{i \in \NZ(f) \setminus S} r_i \ip{e^i}{r} + \sum_{i \in S} r_i
\ip{e^i}{r} \nonumber
%
\intertext{Note that for all $i \in S$, $\ip{e^i}{f - f^S} = 0$, since $f^S$ is
the projection of $f$ on to the space spanned by the variables in $S$. Hence, if
$r^S = f^S - w$, the residual within the space spanned by $S$, then $r = f - f^S
+ r^S$. Thus, we have $\ip{e^i}{r} = \ip{e^i}{r^S} \leq \ltwonorm{r^S}$. Using
this we get, }
%
\ip{r}{r} &\leq \sum_{i \in \NZ(f) \setminus S} r_i \ip{e^i}{r} + K
\ltwonorm{r^S} \sum_{i \in S} |r_i|  \nonumber
%%
\intertext{Now, even by a very crude estimate, $|r_i| = |f_i - w_i| \leq 2 B$,
and hence by the condition in the statement of Claim~\ref{claim:cantaloupe},
that $\ltwonorm{r^S} =\ltwonorm{f^S - w} \leq l^2\Delta^2/(2KW)$ together with
the observation that $\ltwonorm{r}^2 = \ltwonorm{f - w}^2 \geq l^2 \Delta^2$, we
have that,}
%%
\frac{1}{2}\ip{r}{r} &\leq \sum_{i \in \NZ(f) \setminus S} r_i \ip{e^i}{r}
\nonumber
%
\intertext{We now appeal to Lemma~\ref{lemma:amsterdam} (part 1), which shows
that $|r_i| \leq \sqrt{\ip{f - w}{f - w}/\Delta^2} = \ltwonorm{r} / \Delta$, and
conclude that there exists an $i$ for which}
%
|\ip{e^i}{r}| &\geq \frac{\ltwonorm{r} \Delta}{2 k} \nonumber
%
\intertext{The crucial observation is that $|\NZ(f)| \leq k < K$. Let $\beta =
\frac{\ip{r}{e^i}}{\ltwonorm{e^i}^2}$. Finally, Lemma~\ref{lemma:amsterdam}
(part 2) implies that there exists an $i^\prime$ for which $w_{i^\prime}^2 \leq
\ltwonorm{w}^2/K$. We consider the mutation, $w^\prime = w + \gamma e^i -
w_{i^\prime} e^{i^\prime}$ for $\gamma \in [\beta - |\beta|/2, \beta +
|\beta|/2]$. Then, we have}
%%
\ltwonorm{f - (w + \gamma e^i - w_{i^\prime} e^{i^\prime})}^2 &= \ltwonorm{f - (w
+ \gamma e^i)}^2 - 2 w_{i^\prime} \ip{f - (w + \gamma e^i)}{e^{i^\prime}} +
w_{i^\prime}^2 \ltwonorm{e^{i^\prime}}^2 \label{eqn:artichoke}
%%
\intertext{First, we deal with the first term in the above expression and then
bound the latter two.}
%
\ltwonorm{f - (w + \gamma e^i)}^2 &= \ltwonorm{r}^2 - 2 \gamma \ip{r}{e^i} +
\gamma^2 \ltwonorm{e^i}^2 \nonumber \\
&= \ltwonorm{r}^2 - (2 \gamma \beta  - \gamma^2) \ltwonorm{e^i}^2 \nonumber
%
\intertext{As in the proof of Claim~\ref{claim:banana}, for $|\gamma| \in
[|\beta|/2, 3|\beta|/2]$, we have that $- 2 \gamma \beta + \gamma^2 \leq -
3 \beta^2/4$. Hence,}
\ltwonorm{f - (w + \gamma e^i)}^2 &\leq \ltwonorm{r}^2 - \frac{3}{4} \beta^2
\label{eqn:broccoli}
%%
\intertext{To bound the remaining two terms in (\ref{eqn:artichoke}), recall that
$f - w = f - f^S + r^S$ and that $\ip{f - f^S}{e^{i^\prime}} = 0$
(since $i^\prime \in S$). Thus, we get that}
%%
- 2 w_{i^\prime} \ip{f - (w + \gamma e^i)}{e^{i^\prime}} + w_{i^\prime}^2
  \ltwonorm{e^{i^\prime}}^2 &\leq 2 |w_{i^\prime}| |\ip{r^S + \gamma
  e^i}{e^{i^\prime}}| + w_{i^\prime}^2 \ltwonorm{e^{i^\prime}}^2 \nonumber
\intertext{Using the fact that $\ltwonorm{r^S} \leq \gamma$, $|w_i|< \gamma$
(which we'll show below), $|\ip{e^i}{e^{i^\prime}}| \leq 1$ and
$\ltwonorm{e^{i^\prime}} \leq 1$, we get,}
%%
- 2 w_{i^\prime} \ip{f - (w + \gamma e^i)}{e^{i^\prime}} + w_{i^\prime}^2
  \ltwonorm{e^{i^\prime}}^2 &\leq 6 |w_i||\gamma| \label{eqn:cabbage}
%%
\intertext{Recall that $|w_i| \leq \ltwonorm{w}/\sqrt{K}$. Also $\ltwonorm{w}
\leq \ltwonorm{f^S} + \ltwonorm{r^S} \leq 2 \ltwonorm{f^S} \leq 2 \ltwonorm{f}
\leq 2uk$. (Here we've used the fact that $\ltwonorm{r^S}$ is small) Combining,
(\ref{eqn:artichoke}), (\ref{eqn:broccoli}),  (\ref{eqn:cabbage}), the fact that
$|w_i| \leq 2uk/\sqrt{K}$ and $|\gamma| \leq 3 |\beta|/2$, we get}
%%
\ltwonorm{f - (w + \gamma e^i - w_{i^\prime} e^{i^\prime})}^2 &\leq
\ltwonorm{r}^2 - \frac{3}{4} \beta^2 + 18 |\beta|uk/\sqrt{K} \nonumber
%%
\end{align}
Finally, we note that when $K > 5184 (k/\Delta)^4 (u/l)^2$, the above
equation ensures that error drops by at least $\beta^2/4$.  The probability of
choosing a swapping operations is $1/3$, of subsequently choosing the correct
pair is at least $1/(nK)$, and subsequently choosing the correct value of $w_i$
is at least $|\beta|/(2B)$. A simple calculation proves the statement of the
claim.
\end{proof}

\begin{theorem} Let $\Dists$ be the class of $\Delta$-smooth bounded
distributions. Then the class $\lin^k_{l, u}$ is evolvable using the
representation class $\lin^K_{0, B}$, where $K = O((k/\Delta)^4 (u/l)^2)$ and $B
= O(uk/\Delta)$. \end{theorem}
\begin{proof}
The mutator is as defined above. We need to define the value $m$ that determines
the number of mutations chosen to populate the multiset $\Neigh(w, \epsilon)$.


\end{proof}

